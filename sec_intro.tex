%!TEX root = main.tex
%!TEX spellcheck = en_US

\section{Introduction}
\label{sec:intro}

Misconfiguration detection in router configurations is crucial for maintaining the stability, security, and performance of network infrastructures. Whether due to overlooked errors in existing setups or mistakes introduced during configuration changes, misconfigurations can disrupt network behavior, leading to unintended access routes, inefficient routing paths, or even network outages. These disruptions not only compromise the network’s integrity but also result in costly downtimes or security breaches. For instance, a misconfigured access control list (ACL) could allow unauthorized traffic, exposing the network to potential attacks, while a poorly defined routing policy might create traffic loops, degrading network performance.

Current methods for detecting misconfigurations in router configurations can be broadly classified into three categories: automated formal model checkers, data mining-based tools, and Large Language Model (LLM)-based Q\&A models. Model checkers can be either rule-based, including Batfish and rcc, or policy-based, such as Batfish, Minesweeper, Tiramisu, and most of the other network configuration verifiers that have been developed. These verifiers are designed to validate configurations by checking their adherence to predefined rules and policies. While are highly effective, their performance is heavily dependent on the quality of the rules and policies being inputted, which require substantial domain expertise to define. Data mining-based tools, like Diffy, Minerals, and SelfStarter, take a different approach by deriving a universal model through graph-based or data mining techniques, assuming that any deviation from a standard configuration is a misconfiguration. While this method works well for maintaining uniformity across routers, it struggles with context-specific configurations that deviate from the norm for valid reasons. Finally, there are LLM-based Q\&A models that attempt to `understand' the syntax and semantic of configurations by treating misconfiguration detection as a sequential text generation task. These models leveraging transformer-based architectures to process input configurations as sequences of tokens and are pre-trained using large-scale datasets, including general text documents and network configuration texts. This allows the models to learn relationships between tokens (words or phrases) and their contexts through attention mechanisms, a key feature of transformer models.

Despite their strengths, each of these methods has significant weaknesses that limit their effectiveness in accurately detecting router misconfigurations. Model checkers are limited by the need for comprehensive and precise rule and policy definitions, which can be challenging in dynamic network environments. Data mining-based tools lack the ability to understand the underlying semantics of configurations, making them prone to false positives in cases where deviations are intentional and context-specific. LLM-based Q\&A models, while powerful, face challenges in how they are prompted. The current practice of partition-based prompting, where configuration files are divided into isolated sections for analysis, fails to capture the interdependencies and contextual nuances spread across the configuration. This results in a reduced ability to detect complex misconfigurations that depend on a holistic understanding of the entire configuration file. To address these limitations, there is a need to design a framework that automates the process of context extraction and optimally structures the prompts for LLMs. Such a framework would enhance the accuracy of router misconfiguration detection by ensuring that all relevant context is considered, enabling more precise and reliable verification.

In this paper, we introduce a new framework, Context-Aware Iterative Prompting (\sysname{}), designed to automate context extraction and optimize prompting to LLMs for accurate router misconfiguration detection. We identify three key challenges in realizing this framework and provide solutions to address them:

\mypara{Challenge 1 - Efficient and Accurate Context Mining}
Network configuration files are often lengthy and complex, with interrelated lines that require careful context extraction. The challenge lies in efficiently identifying and extracting relevant context to reduce computational costs and avoid introducing irrelevant information that could impair the accuracy of misconfiguration detection. \textbf{Solution}: We address this by leveraging the hierarchical structure of configuration files, modeling them as trees where each line is a unique path from root to parameter value. This allows us to systematically mine three types of context—neighboring, similar, and referenceable configurations—ensuring that the LLM receives the most relevant information for accurate analysis.

\mypara{Challenge 2 - Parameter Values Ambiguity in Referenceable Context}
Configurations contain both pre-defined and user-defined parameter values, with the latter requiring context for proper interpretation. When extracting referenceable context, misidentifying these can lead to irrelevant context mining and increased computational overhead. \textbf{Solution}: To accurately differentiate these values, we implement an existence-based method within the configuration tree, identifying user-defined values by their presence as intermediate nodes in other paths. Additionally, we use a majority-voting approach to resolve ambiguities, ensuring consistency in how values are treated across different contexts.

\mypara{Challenge 3 - Managing Context Overload in Prompts}
Even with precise context extraction, the volume of relevant information can be extensive, risking prompt overload when fed into LLMs. Excessive context can dilute relevance, reduce coherence, and degrade the model’s performance, necessitating strategies to manage and streamline the input effectively. \textbf{Solution}: Our framework mitigates this by allowing the LLM to iteratively request specific types of context, refining the prompt based on the model’s feedback. This approach prevents overload and enhances the model’s ability to focus on the most relevant information, leading to more precise misconfiguration detection.

% We evaluate our framework through three distinct case studies: (1) synthetic misconfigurations introduced into collected router configurations, covering three broad categories of misconfigurations, (2) real-world misconfigurations found in actual network router configurations, and (3) exhaustive misconfiguration detection across multiple router configurations within a medium-scale campus network. Our evaluation results demonstrate that \sysname{} consistently outperforms or matches the accuracy of existing methods across all scenarios.

We evaluate our framework through two distinct case studies: (1) targeted change verification: using configuration snapshots from real networks, we introduce synthetic modifications to simulate potential changes that network operators may make (both correct and incorrect). We demonstrate that the model accurately infers and detects these changes. (2) comprehensive snapshot verification: we exhaustively run \sysname{} on configuration snapshots from a medium-sized campus network to detect overlooked misconfigurations in real-world environments. Our evaluation shows that \sysname{} consistently outperforms or matches the accuracy of existing methods across all scenarios.

% Automated Rule-based Verifiers (Batfish): Rule-based verifiers like Batfish are designed to validate network configurations by checking their conformance to predefined rules and policies. These tools excel at ensuring that configurations adhere to syntactical correctness. However, their effectiveness is heavily dependent on the quality and comprehensiveness of the manually defined rules, which requires significant domain expertise for each specific type of configuration. This reliance on domain knowledge poses a challenge in dynamically evolving network environments where configurations frequently change. Moreover, while Batfish can effectively detect syntax errors and rule violations, it lacks the ability to fully comprehend the semantics and interactions between different configuration components. This limitation means that more complex issues, such as missing definitions or subtle misconfigurations resulting from component interactions, may go undetected.
% Data Mining-based Tools (Diffy): Data mining-based tools, such as Diffy, approach configuration verification by deriving a universal model, often through graph-based or data mining techniques. These tools assume that any deviation from an established standard configuration represents a misconfiguration. While this assumption works well for identifying issues related to uniformity across routers—where consistency is crucial—it falls short in scenarios where routers require specific, context-driven changes. These tools are particularly weak in detecting syntax or range-related issues because they treat configurations purely as data or graphs, without understanding the underlying semantics. Consequently, while Diffy can identify deviations from a norm, it struggles with configurations that deviate from the standard for valid, context-specific reasons.
% LLM-based Q\&A Models: Large Language Models (LLMs), trained on vast datasets that include networking configuration data, have recently gained attention as a promising approach for configuration verification. These models can be used either in their generic form, leveraging their pre-existing understanding of configuration languages, or fine-tuned for specific configuration types to improve accuracy and specialization. LLM-based Q\&A models overcome many of the limitations of traditional data mining or rule-based verifiers by understanding both the syntax and semantics of configurations. However, a significant challenge remains in how these models are prompted. Current practices often involve partition-based prompting, where the configuration file is divided into isolated sections, and each section is independently analyzed by the model. This approach fails to account for the inherent dependencies and context spread across the configuration file. Network configurations, like many complex system configurations, contain interdependent elements and contextual nuances that are crucial for accurate verification. If these dependencies and contexts are not properly extracted and presented to the model in a guided and precise manner, the model's ability to detect sophisticated, context-relevant misconfigurations is compromised. Unfortunately, there is no readily available tool designed to extract and structure these contexts in an LLM-friendly manner, leading to reduced accuracy in identifying complex misconfigurations that depend on a holistic understanding of the configuration file.


% Challenge 1: Efficient and accurate context mining: Configuration files are often lengthy and complex, consisting of multiple interrelated lines that define various aspects of a system's behavior. When analyzing a specific configuration line, it is crucial to efficiently identify and extract the relevant context to reduce the computational burden on large language models (LLMs) and minimize the risk of introducing irrelevant information that could degrade the accuracy of misconfiguration detection. This task is challenging because these configurations are typically written in a human-readable format, and automating the process of context extraction requires a deep understanding of the hierarchical and interconnected nature of the configuration data.

% Challenge 2: Referenceable parameter Value Ambiguity: In network configurations, parameter values can broadly be categorized into two types: pre-defined values and user-defined values. Pre-defined values are those that are built into the configuration language itself, such as boolean flags (True vs. False) or access control decisions (Allow vs. Deny). These values are universally understood by the configuration parser and typically do not require any additional definition or context within the configuration file. On the other hand, user-defined values are customizable by the network administrator and can vary widely depending on the specific requirements of the network. These include values such as IP addresses, timeout intervals, VLAN IDs, and other numeric or alphanumeric identifiers. However, there rarely exists documentation explicit specifying these types and requires a lot of manual examination, which is hard to scale.
% When performing context extraction as discussed in Challenge 1, it is crucial to distinguish between pre-defined and user-defined parameter values for several reasons:
% Avoiding Irrelevant Context Mining: If the context mining process does not differentiate between pre-defined and user-defined values, it risks introducing irrelevant or misleading information. For instance, consider a configuration parameter with the pre-defined value Allow. In a network configuration, Allow could be used in various contexts, such as: Allowing traffic through a firewall rule (FirewallRule → Action = Allow), Enabling a specific feature (FeatureControl → Enable = Allow), Granting user permissions (UserPermissions → AccessLevel = Allow). Mining for distant context based on the value Allow could yield unrelated configurations, such as those controlling user access rather than network traffic, thereby contaminating the context pool with irrelevant data. This contamination can lead to incorrect or suboptimal decisions by the LLM, as the model may be presented with a context that is not relevant to the specific configuration line under examination. Similarly, consider the pre-defined value True. In network configurations, True could be used to: Enable a particular feature (FeatureX → Enabled = True), Set a flag within a routing protocol (OSPF → PassiveInterface = True), Indicate a boolean condition in a script (ScriptCondition → IsValid = True). Mining context based on the value True could result in the extraction of unrelated configuration sections that only share the use of the boolean value but pertain to entirely different functionalities. Without distinguishing between pre-defined and user-defined values, the context mining process might erroneously group together configurations that pertain to entirely different aspects of network operation, thereby confusing the LLM and reducing the accuracy of its inferences.
% 2. Reducing Computational Costs: LLMs can be computationally expensive, particularly when processing large-scale network configurations with numerous interdependent components. By distinguishing between pre-defined and user-defined values, we can optimize the context extraction process, ensuring that only relevant and necessary contexts are passed to the model. This not only improves the accuracy of misconfiguration detection but also reduces the computational overhead associated with processing extraneous information. For instance, consider the case of mining context for a user-defined IP address. Suppose a configuration line specifies Interface → IPAddress = 192.168.1.1. In this case, the relevant context might include other configurations that reference this specific IP address, such as firewall rules or routing entries. However, if the mining process mistakenly treats Allow or True in the same way as user-defined values, it might pull in unrelated contexts from access control lists or feature flags, resulting in unnecessary processing and potentially higher costs.
% Challenge 3: Context Overload in Prompt
% Even with the precise context extraction mechanism we have, the context extracted can sometime be very long. This can be problematic as LLM tend to have the problem of context (or prompt) overload when a large amount of context or information is provided in a single prompt, the model may struggle to maintain focus, leading to issues like: (1) Dilution of Relevance: The model may lose track of the most relevant parts of the prompt, leading to responses that are less accurate or on target. (2) Loss of Coherence: The model might generate responses that are disjointed or fail to coherently address the query, as it attempts to process and incorporate all the given information. (3) Decreased Performance: The overall quality of the model’s output may degrade because it has to handle too much information at once, which can overwhelm its processing capabilities.


% Solution 1: To address this challenge, we can leverage two key observations:
% (1) Hierarchical Structure of Configuration Files: Configuration files are predominantly written in a structured, hierarchical format that can be effectively modeled as a tree. In this tree, each node represents a specific configuration element, and edges represent the relationships between these elements. The root node corresponds to the broadest configuration category, while intermediate nodes represent nested configuration sections or subcategories. The leaf nodes (or parameter nodes) are the configurable parameters, with the associated values (i.e., parameter value) representing the final configuration state. Here, a unique configuration line corresponds to a unique path in the tree, beginning at the root and terminating at a parameter node with an assigned parameter value.
% (2) Contextual Relevance from Different Aspects: The context related to any given configuration line can be categorized into three types based on its position and relation within the tree structure: (a) referenceable configurations: These configurations provide additional information or definitions related to the parameter value of the configuration line under examination. In tree terminology, this involves finding paths where the parameter value of the current configuration line appears as an intermediate node in other configuration paths. These distant paths serve to define or elaborate on the parameter value, providing a broader understanding of its role within the configuration. (b) adjacent configurations and their referenceable configuration: Adjacent configurations are those that share a common sub-path with the configuration line under examination, specifically up to the parent node of the parameter node. For example, if the configuration path is A→B→C→D = 10, adjacent configurations would be other paths stemming from node C. These paths offer insights into how related parameters are configured within the same section or neighborhood of the configuration tree, and we also look at referenceable configurations containing context to these adjacent configurations, (3) similar configurations: These are configurations that share the same root node and parameter node but may differ in their intermediate nodes or parameter values. Such configurations are important for understanding how similar parameters are configured across different contexts within the system. For instance, comparing paths that have the same root and parameter node (A and D) but different intermediate nodes (A → B → C1 → D = 20 vs. A → B → C2 → D = 30) can reveal variations in how the parameter D is applied in different scenarios.
% Building on the hierarchical tree model, we can automate the context mining process by implementing the following steps:
% Referenceable Configuration Mining: Identify configuration paths where the parameter node of the current line appears as an intermediate node. These paths are likely to contain definitions or supplementary information relevant to the parameter value.
% Adjacent Configuration and their Referenceable Configuration Mining: Extract paths that share a common sub-path with the current configuration line up to the parent node of the configurable parameter node. This reveals the configuration of related parameters in the same section. We follow the same logic as above to mine the referenceable configuration to these adjacent configurations.
% Similar Configuration Mining: Locate paths that have the same root and parameter nodes but differ in intermediate nodes or parameter values. This helps in understanding the parameter's application across various contexts within the configuration.
% By structuring the context mining process in this manner, we can efficiently and accurately extract the most relevant information for any given configuration line, enhancing the performance and accuracy of LLM-based configuration verification tools. This approach ensures that the LLM receives a well-curated set of contextually relevant data, improving its ability to detect sophisticated misconfigurations that depend on a comprehensive understanding of the configuration file.

% Solution 2: Existence and Majority-Voting Based Differentiation: 
% To accurately differentiate between pre-defined and user-defined parameter values within network configuration files, we make the following observations and propose a solution based on existence checks within the configuration tree structure and majority-voting mechanisms.
% Existence-Based Differentiation:
% Observation: User-defined values typically carry distant contextual information and often have associated definitions or explanations elsewhere in the configuration file. For example, if a parameter value represents a specific, customized import policy, the configuration file should contain other lines that further explain or define this policy's intended behavior.
% Translation to Tree Structure: In the hierarchical tree model of the configuration, if a parameter value is user-defined, it is likely to appear as an intermediate node in other configuration paths, rather than solely as a parameter value. This presence indicates that the value is being referenced or elaborated upon elsewhere in the configuration. Conversely, if no such paths exist, the value is most likely pre-defined, meaning it requires no further explanation or context.
% Special Cases: Some parameters accept flexible values, such as ranges or numeric values, which may not have explicit definitions within the configuration. These values are treated as pre-defined because they are standardized and do not typically require additional context.
% Challenges with Existence-Based Differentiation
% Finer-Grained Ambiguity in Parameter Values: The same parameter value can function as user-defined in one context and pre-defined in another. For instance, the value 1000 used in a timeout setting might be pre-defined (requiring no further definition), while 1000 used as an import policy name or group name might be user-defined and require contextual elaboration. This distinction is crucial because identical values can have different meanings based on the configuration parameter they are associated with.
% Majority-Voting for Consistency:
% Observation: To address the ambiguity, we avoid assigning a universal type to a parameter value across all configurations. Instead, we determine the type (pre-defined or user-defined) based on the specific combination of the configuration parameter and its associated value. This approach ensures that the same value is correctly interpreted according to its context.
% Implementation of Majority Voting: For each configuration parameter, we analyze all associated values to determine whether they are predominantly pre-defined or user-defined. If the majority of values under a particular parameter are identified as user-defined (i.e., they have associated contextual definitions elsewhere in the configuration), we classify all values under that parameter as user-defined. Conversely, if the majority are pre-defined, we classify the parameter accordingly. This approach leverages the principle that configuration parameters should exhibit uniformity in the type of values they accept, thereby maintaining consistency across the configuration.
% To address the challenge of maintaining model focus amid excessive context, we observe that the model performs better when prompted with concise and highly relevant information tailored to the specific type of misconfiguration under scrutiny. For example, when detecting syntax misconfigurations, distant configurations might be less valuable compared to similar or referenceable configurations. The latter provides a clearer contextual template for how similar configuration lines should be syntactically structured. On the other hand, for issues like dependency conflicts, different types of context—such as definitions or dependencies—might be more relevant.
% This observation leads us to design a framework that leverages the model’s ability to discern and request the type of context it requires for making better decisions. The framework operates in a sequential and interactive manner:
% Initial Prompting: We start by presenting the model with the configuration line under review, along with specific instructions regarding the type of misconfiguration we want it to detect (e.g., syntax errors, dependency conflicts).
% Contextual Options: We provide the model with options to request additional context, such as:  Distant configurations with relevant definitions. Adjacent configurations that might influence the current line. Similar configuration lines that serve as templates or references. 
% Iterative Refinement: We allow the model to decide whether the initial prompt contains sufficient information. If the model indicates that more context is needed, we iteratively supply the requested context, refining the prompt based on the model's feedback.
% This approach not only prevents context overload but also empowers the model to actively participate in the decision-making process by identifying and requesting the specific information it needs. Consequently, this leads to more accurate and focused detection of misconfigurations, as the model is guided to hone in on the most relevant aspects of the configuration under review.

